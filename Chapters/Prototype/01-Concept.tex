%-----------------------------------------------
% Dateiname: Concept.tex
% Autor    : Stefano Kowalke <blueduck@gmx.net>
% Lizenz   : BSD
%-----------------------------------------------
\section{Konzeption des Prototypen}
\label{prototype:sec:concept}
Bevor mit der Implementierung des Prototyps begonnen werden konnte, wurden die zu erreichenden Ziele definiert.

\begin{itemize}
\item  Der Prototyp ist eine normale Extension, die über das \textit{Install Tool} installierbar ist. Dies ist notwendig, da bereits bei der Installation das zu nutzende \gls{dbms} auswählbar sein muß. Er ist gegenfalls ohne größeren Aufwand in eine Systemextension umwandelbar.

\item Der Prototyp unterstützt die alte Datenbank \gls{api}, damit TYPO3 CMS und externe Extensions weiterhin funktionieren.
\item Der Prototyp unterstützt MySQL als \gls{dbms}.
\item Die Methodennamen der neuen \gls{api} folgen den TYPO3 \gls{cgl}.
\item Die Erstellung der Basisdatenbank erfolgt durch Doctrine DBAL und nutzt dessen abstaktes Datenbankschema.
\item Der Prototyp führt eine \textit{Fluent Query Language} ein, damit auf die manuelle Formulierung von SQL Anfragen verzichtet werden kann.
\item Der Prototyp nutzt intern \textit{Prepared Statements}.
\item Der Prototyp erhält den Extensionkey \textit{doctrine\_dbal}.
\end{itemize}

Diese Anforderungen konnten anschließend in einzelne Teilaufgaben zusammengefasst werden:

\begin{enumerate}
\item Erhöhung der Testabdeckung der vorhandenen Datenbank \gls{api}
\item Erstellen der Grundstrukutur des Prototypen
\item Implementation einer Fluent \gls{api}
\item Umbau von TYPO3 CMS auf die \gls{api} des Prototypen
\end{enumerate}
