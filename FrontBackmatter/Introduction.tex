%-----------------------------------------------
% Dateiname: Introduction.tex
% Autor    : Stefano Kowalke <blueduck@gmx.net>
% Lizenz   : BSD
%-----------------------------------------------
\chapter{Einleitung}
\label{ch:intro}
Als sich auf den Developer Days 2006 das Entwicklerteam für einen Nachfolger der eben erst erschienen \gls{cms} TYPO3 Version 4.0 formierte[News T3Org], war wohl keinem der dort Anwesenden klar wohin die Reise gehen würde – ging man anfänglich noch von einem Restrukturierung (engl. Refactoring) der schon vorhandenen Codebasis aus.

In der Konzeptionsphase kristallisierte sich immer mehr heraus, dass es damit nicht getan sein würde. Der Nachfolger mit dem Arbeitstitel ``Phoenix'' sollte nicht nur den zukünftigen Anforderungen des Web standhalten, sondern die Position der Version 4.0 weiter ausbauen. Das Entwicklerteam um Chefentwickler Robert Lemke entschloss sich die Version 5.0 des Systems komplett neu zu schreiben [Quelle anfügen] und merkte dabei, dass Entwickler bei der Programmierung von Webanwendungen immer wieder mit den gleichen Problemen wie Routing, die Erstellung und Validierung von Formularen, Login von Benutzern oder dem Aufbau einer Verbindung zur Datenbank konfrontiert werden.

Die Idee eines – von dem Content-Management-System – unabhängigen PHP Frameworks war geboren und wurde zunächst auf den Namen FLOW3 getauft. Dieses Framework sollte die spätere Basis für TYPO3 5.0 bilden und all die oben beispielhaft angeführten wiederkehrenden Aufgaben übernehmen. Die Version 5.0 von TYPO3 sollte lediglich eins von vielen Packages darstellen mit denen FLOW3 erweitert werden kann. Vielmehr wurde es als eigenständiges ``Webapplication Framework'' konzipiert und umgesetzt, so dass es auch ohne ein \gls{cms} betrieben werden kann und auch wird. [Quelle zu Rossmann einfügen].

Schon in einer recht frühen Entwicklungsphase hat man sich dem Thema Persistenz gewidmet, die zunächst noch als ``\gls{jcr}'' in PHP implementiert [Quelle einfügen], jedoch später wegen zu vieler Probleme bei der Portierung der Java Spezifikation JSR-170 [stimmt die Nummer?] nach PHP [Quelle Blogeintrag von Karsten] durch eine eigene Persistenzschicht ersetzt wurde. Im weiteren Verlauf der Entwicklung kam man von dieser Idee wieder ab, da das Bereitstellen und die Pflege einer eigenen Datenbankschnittstelle zu aufwändig sei und andere Projekte wie Doctrine oder Propel schon fertige Lösungen anboten. Schlußendlich entschied man sich für die Integration von Doctrine als Persistenzschicht. [Quellen zu Github einfügen]

Für die Anwender stellt sich bei einem Versionssprung stets die Frage, ob eine Migration von der alten zur neuen Version möglich ist und mit wieviel Aufwand dies verbunden sein würde. Diesen Bedenken folgend trafen sich die Kernentwickler beider Teams 2008 in Berlin, um die Routemaps beider Projekte in Einklang zu bringen. Als ein Ergebnis dieses Treffens wurde das ``Berlin Manifesto'' [Quelle angeben] bekanntgegeben, welches mit kappen Worten feststellt\footnote{Mittlerweile wird TYPO3 Neos innerhalb der Community nicht mehr als der Nachfolger von TYPO3 \gls{cms} angesehen. Es stellt lediglich – wie TYPO3 Flow – ein weiteres Produkt innerhalb der TYPO3 Familie dar. [Quellen angabe]}
:
\begin{shadequote}[l]{Die TYPO3 Kernentwickler}
	\begin{itemize}
		\item TYPO3 v4 continues to be actively developed
		\item v4 development will continue after the the release of v5
		\item Future releases of v4 will see its features converge with those in TYPO3 v5
		\item TYPO3 v5 will be the successor to TYPO3 v4
		\item Migration of content from TYPO3 v4 to TYPO3 v5 will be easily possible
		\item TYPO3 v5 will introduce many new concepts and ideas. Learning never stops and we'll help with adequate resources to ensure a smooth transition
	\end{itemize}
\end{shadequote}

An der Umsetzung wurde sofort nach dem Treffen begonnen, indem Teile des FLOW3 Frameworks nach TYPO3 Version 4.0 zurück portiert und unter dem Namen \emph{Extbase} als Extension veröffentlicht wurden. Es erfüllt zu gleichen Teilen die Punkte 3 und 6 des Manifests, da es die neuen Konzepte aus FLOW3 der Version 4.0 zur Verfügung stellt und somit gleichzeitig diese Version näher an die Technologie des Frameworks heranführt.

Die Aufgabe von Extbase besteht darin ein \gls{api} bereitzustellen, mit denen Entwickler von Extensions auf die internen Ressourcen und Funktionen von TYPO3 \gls{cms} zugreifen und das System somit nach eigenen Wünschen und Anforderungen erweitern können, ohne den Code des \gls{cms} selbst verändern zu müssen. Es ist als vollständiger Ersatz der bis dahin angebotenen PI-Base \gls{api} [LINK ZU PI BASE] konzipiert worden, wobei es aktuell noch möglich ist sich für einen der beiden Ansätze zu entscheiden.

Extbase führt per Definition einige – bis dahin in TYPO3 v4 unbekannte – Programmierparadigmen ein. Als größter Unterschied zu dem PI-Based Ansatz ist hier sicherlich das \gls{mvc} Pattern zu nennen. Dabei werden die Daten im Model vorgehalten, der View gibt die Daten aus und der Controller steuert die Ausgabe der Daten. Das Model ist unabhängig von der View, was bedeutet, dass die gleichen Daten auf verschiedene Weise ausgegeben werden können. Man denke hier an Meßdaten, die zum einen als Tabelle über einer Listview dargestellt werden können oder als Diagramme mit einer entsprechenden View.

Das Model – eine herkömmliche PHP Klasse – wird dabei von Extbase automatisch auf die Datenbank abgebildet, so dass ein Objekt eine Zeile darstellt und dessen Eigenschaften als Spalten der Tabelle interpretiert werden. Diese Technik wird als Objektrelationale Abbildung (engl. \gls{orm}) genannt. Das zum Einsatz kommende \gls{orm} ist Bestandteil der oben erwähnten selbstgeschriebenen Persistenzschicht von FLOW3, da Extbase zu der Zeit rückportiert wurde, als diese bei FLOW3 im Einsatz war.

Obwohl Extbase beständig weiterentwickelt wird und es der Wunsch der Community ist, die in darin verwendete Persistenzschicht gegen Doctrine 2 auszutauschen, was sich in Form von Posts auf der Mailingliste [Ausdruck und Link der Quelle einfügen] oder in Prototypen ausdrückt [Link der Quelle einfügen], ist dies bis heute noch nicht realisiert worden. Der Chefentwickler von Doctrine, Benjamin Eberlei, hat gegenüber dem Autor in einer persönlichen Korrespondenz die unterschiedlichen Ansätze beider Projekte wie folgt zum Ausdruck gebracht:
\begin{shadequote}{Benjamin Eberlei per E-Mail vom 17.12.13 00:12}
	(\ldots) Doctrine nutzt das Collection interface, Extbase SplObjectStorage.Doctrine Associationen funktionieren semantisch anders als in Extbase, z.B. Inverse/Owning Side Requirements.
	Typo3 hat die Enabled/Deleted flags an m\_n tabellen, sowie das start\_date Konzept. Das gibts in Doctrine \gls{orm} alles evtl nur über Filter \gls{api}, aber vermutlich nicht vollständig abbildbar.
	Das betrifft aber alles nur das \gls{orm}, das Doctrine \gls{dbal} hinter Extbase zu setzen ist ein ganz anderes Abstraktionslevel.
\end{shadequote}
[Email in Literatur anhängen und als Quelle anfügen]

Zum jetzigen Zeitpunkt wird die \gls{dbal} in TYPO3 durch eine Systemextension [Glossareintrag] bereitstellt, die auf der externen Bibliothek AdoDB basiert, welche jedoch Anzeichen des Stillstands aufzeigt und davon ausgegangen werden kann, dass das Projekt nicht weiterentwickelt wird. [Linkt zu SourceForge]

Anhand dieser Fakten wird ersichtlich, dass die Integration von Doctrine erstrebenswert ist, da dadurch die Abhängigkeit zu dem inaktiven Projekt AdoDB aufgelöst werden kann. Da jedoch eine Integration von Doctrine \gls{orm} in Extbase nicht in der gegebenen Zeit, die für die Bearbeitung der Thesis zur Verfügung steht, zu realisieren ist, wurde der Fokus stattdessen auf die Integration von Doctrine \gls{cms} in TYPO3 gelegt, wodurch nicht nur Extbase von den Möglichkeiten eines \gls{dbal} profitieren kann, sondern der gesamte Core und somit alle Extensions die noch nicht mit Extbase erstellt worden sind.

Ferner wird durch diesen Ansatz eine stabile Basis zu Verfügung gestellt, auf der eine zukünftige Integration der \gls{orm} Komponente von Doctrine in Extbase aufbauen kann.

Ziel dieser Thesis ist es einen funktionierenden Prototypen zu entwickeln, der zum einen aus einer Extension besteht, die für die Integration von Doctrine \gls{dbal} zuständig ist und zum anderen aus einem modifizierten TYPO3, welches die neue \gls{api}, die mit der Extension eingeführt wird, beispielhaft benutzt.

Im ersten Teil werden die eingesetzten Werkzeuge vorgestellt. Es wird erklärt warum diese und nicht andere eingesetzt worden sind und wie diese in Hinblick auf die Aufgabenstellung benutzt wurden.

Der zweite Teil beschreibt die praktische Umsetzung und schließt mit einer Demonstration wie der Prototyp getestet werden kann.

Teil drei gibt einen Ausblick auf die weitere Verwendung des Quellcodes und des Prototypen, während Teil vier mit einem Fazit schließt.

Im Anhang befindet sich neben den obligatorischen Verzeichnissen für Literatur und Abbildungen ein Glossar, sowie das Abkürzungsverzeichnis.
