%-----------------------------------------------
% Dateiname: Definitions.tex
% Autor    : Stefano Kowalke <blueduck@gmx.net>
% Lizenz   : BSD
%-----------------------------------------------

%-----------------
% Abkürzungen
%-----------------
% http://tex.stackexchange.com/questions/8946/how-to-combine-acronym-and-glossary
\newglossaryentry{dbal}
{
	type=\acronymtype,
	name={DBAL},
	description={Database Abstraction Layer},
	first={Database Abstraction Layer (DBAL)},
	see=[Glossary:]{dbalg}
}

\newglossaryentry{bafög}
{
	type=\acronymtype,
	name={BAföG},
	description={Bundesausbildungsförderungsgesetz},
	first={Bundesausbildungsförderungsgesetz (BAföG)}
}

\newglossaryentry{cms}
{
	type=\acronymtype,
	name={CMS},
	description={Content-Management-System},
	first={Content-Management-Sytem (CMS)}
}

\newglossaryentry{orm}
{
	type=\acronymtype,
	name={ORM},
	description={Object-relational mapping},
	first={Object-relational mapping (ORM)}
}

\newglossaryentry{jcr}
{
	type=\acronymtype,
	name={JCR},
	description={Content Repository for Java Technology API},
	first={Content Repository for Java Technology API (JCR)}
}

\newglossaryentry{mvc}
{
	type=\acronymtype,
	name={MVC},
	description={Model-View-Controller},
	first={Model-View-Controller (MVC)}
}

\newglossaryentry{api}
{
	type=\acronymtype,
	name={API},
	description={Application Programming Interface},
	first={Application Programming Interface (API)}
}
%-------------
% glossareinträge
%-------------
\newglossaryentry{dbalg}
{
	name={dbal},
	description={a very long description of of what is dbal}
}
