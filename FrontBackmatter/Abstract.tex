%-----------------------------------------------
% Dateiname: Abstract.tex
% Autor    : Stefano Kowalke <blueduck@gmx.net>
% Lizenz   : BSD
%-----------------------------------------------
\chapter{Abstract}
\label{ch:abstract}
Webanwendungen werden häufig um ein Datenbankmanagementsystem herum entworfen. In der Vergangenheit war dies oft MySQL. Um eine Webanwendung aus der Abhängigkeit zu einem spezifischen Datenbankmanagementsystem zu lösen, kann die, von PHP mitgelieferte, Datenbankabstraktionsschicht PDO genutzt werden. Einen Schritt weiter geht Doctrine DBAL, welches – auf PDO aufbauend – eine einheitliche Schnittstelle zu weiteren Datenbankmanagementsystemen bereitstellt. Doctrine DBAL ist zudem die Grundlage für Doctrine ORM - ein Framework zur Objekt-relationalen Abbildung von Objekten auf eine Datenbank, das in TYPO3 Flow und TYPO3 Neos eingesetzt wird. Diese Arbeit demonstiert wie durch die Integration von Doctrine DBAL in das Content-Manangement-System TYPO3 CMS die Abhängigkeit zu MySQL entfernt werden kann.
