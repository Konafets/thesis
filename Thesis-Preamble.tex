%-----------------------------------------------
% Dateiname: Thesis-Preamble.tex
% Autor    : Stefano Kowalke <blueduck@gmx.net>
% Lizenz   : BSD
%-----------------------------------------------

%----------------------------------
% Dokumentenklasse DINA4 einseitig
%----------------------------------
\documentclass[
	fontsize    = 11pt,           % Die Schriftgröße
	twoside     = false,          % scrbook hat per Default ein Zwei-Seitenlayout
	parskip     = full,           % Steuert die Absätze. http://www.rrzn.uni-hannover.de/fileadmin/kurse/material/latex/scrguide.pdf Tabelle 3.7
	headsepline,                  % Fügt eine Trennungslinie in den Seitenkopf
	footnotes   = multiple,       % Fügt ein Komma zwischen den Indexzahlen bei aufeinanderfolgende Fußnoten ein
	numbers     = noendperiod     % Keinen Punkt der letzten Gliederungsebene in der Überschrift  -> 1.2.1 statt 1.2.1.
]{scrbook}

\PassOptionsToPackage{
	%layout,
	drafting,
    eulerchapternumbers,
	eulermath,
	colophon,
	bettertable,
    %minionpro,
%	dottedtoc,
}{thesis}


%===============
% Pakete laden
%===============
\usepackage{fontspec}                      % Wird von LuLaTeX benöigt und löst "fontenc" ab.
\usepackage{polyglossia}                   % Wird von LuLaTeX benöigt und löst "babel" ab.
\usepackage[german=quotes]{csquotes}       % Anführungszeichen global im Dokument steuern. Paket wird von "polyglossia" empfohlen.
\usepackage[
	backend=biber,                         % Benutzer biber zur Erstellung
	bibwarn=true,                          % Warne, wenn das BiTex Format falsch ist
	autolang=other,
	style=authoryear,
	bibstyle=iso-authoryear,
]
{biblatex}                                 % Nutze Biblatex zur Erstellung des Literaturverzeichnis
\addbibresource{Bib/Bibliography.bib}      % Die Literatureinträge
\usepackage{minted}                        % Sourcecode Highlighting. Dieses Package benötigt Python und Pygments 1.5. Version 1.6 macht Probleme mit gerade Anführungszeichen (') - es stellt sie als normale Anführungszeichen dar.
\usepackage[punct-after=true]{fnpct}       % Ermöglicht das Setzen der Indexzahlen der Fußnoten hinter dem Punkt oder Komma. Hier ist es dafür gedacht die Option footnotes=multiple von KOMA wiederherzustellen, die durch das Hyperref Paket kaputt gegangen ist.
\usepackage{hyperref}                      % Stellt Links in Schreibmaschinenschrift dar und legt einen Link über den Text.
                                           % Dieses Package sollte als letztes aufgerufen werden, da es Problem mit Anderen geben könnte
\usepackage[
    xindy={language=german,codepage=din5007-utf8}, % Ruft Xindy zum Erstellen des Index in der deutschen Version auf
    toc,                                   % Fügt die Glossare dem Inhaltsverzeichnis zu
    acronym,                               % Erstellt ein neues Glossar mit dem Label "acronym"
    nonumberlist,                          % Fügt die Seitenzahlen hinzu, auf denen der Eintrag vorkommt, nicht hinzu
    nopostdot                              % Entferne den Punkt am Ende der Definition
    ]{glossaries}                          % Erstellt Glossar und Abkürzungsverzeichnis. Laut der Dokumentation ist es ausdrücklich notwendig, dass es nach dem Package hyperref eingebunden werden muß
\makeglossaries                            % Anweisung das Glossar zu erstellen

%=================================================
% Angaben zur Arbeit wie Titel und Name des Autor
%=================================================
\newcommand{\myTitle}{Integration der Datenbank-Abstraktionsschicht Doctrine2\xspace}
\newcommand{\myTitleSecondLine}{ in das Content-Management-System TYPO3\xspace}
%\newcommand{\mySubtitle}{Put your subtitle here\xspace}
%\newcommand{\myDegree}{Put your degree here\xspace}
\newcommand{\myName}{Stefan Kowalke\xspace}
\newcommand{\myEMail}{<stefan.kowalke@stud.fh-flensburg.de>\xspace}
\newcommand{\myMatricleNumber}{485366\xspace}
\newcommand{\myProf}{Prof. Dr. Hans-Werner Lang\xspace}
\newcommand{\myOtherProf}{Dipl. VK Tobias Hiep\xspace}
%\newcommand{\mySupervisor}{Put name here\xspace}
\newcommand{\myUni}{\uppercase{\large Fachhochschule Flensburg}\xspace}
\newcommand{\myDepartment}{Angewandte Informatik\xspace}
%\newcommand{\myFaculty}{Put data here\xspace}
\newcommand{\myMajor}{Medieninformatik\xspace}
\newcommand{\myLocation}{Flensburg\xspace}
\newcommand{\myTime}{März 2014\xspace}
%\newcommand{\myVersion}{version 4.1\xspace}


%----------------
% Renew commands
%----------------
%\renewcommand*{\multfootsep}{,\nobreakspace}  % Fügt bei den hochgestellten Indexzahlen von Fußnoten ein Leerzeichen nach dem Komma ein
\deffootnote{1em}{1em}{\thefootnotemark\ }    % Setzt die Indexzahlen in den Fußnoten etwas entfernt vom Text

%---------------------------------------------------------------
% Renew the citation style from parenthesis to square brackets:
%---------------------------------------------------------------
% (Popel 2007, S. 59–63) -> [Popel 2007, S. 59–63]
% http://tex.stackexchange.com/questions/16765/biblatex-author-year-square-brackets
%---------------------------------------------------------------
\makeatletter
\newrobustcmd*{\parentexttrack}[1]{%
  \begingroup
  \blx@blxinit
  \blx@setsfcodes
  \blx@bibopenparen#1\blx@bibcloseparen
  \endgroup}

\AtEveryCite{%
  \let\parentext=\parentexttrack%
  \let\bibopenparen=\bibopenbracket%
  \let\bibcloseparen=\bibclosebracket}
\makeatother

%------------------
% Eigene Kommandos
%------------------


%---------------------
% Spracheinstellungen
%---------------------
\setdefaultlanguage[spelling=new]{german}   % Die Sprache muß vor dem Einbinden von dem Blindtextpackage eingestellt werden
\usepackage{blindtext}                      % Erstellt schnell und einfach Blindtexte mit \Blindtext. Wird ausnahmsweise hier eingebunden


%-------------------
% Linkkonfiguration
%-------------------
\hypersetup
{
	pdftitle       = {\myTitle \myTitleSecondLine},
	pdfauthor      = {\myName},
	pdfsubject     = {\myTitle \myTitleSecondLine},
	pdfcreator     = {\myName},
	pdfkeywords    = {typo3} {dbal} {doctrine} {mysql} {postgres},
	linktoc        = all,
	colorlinks     = true,
	linkcolor      = black,
	citecolor      = black,
	filecolor      = black,
	urlcolor       = blue,
}

%----------
% Grafiken
%----------
\graphicspath{ {gfx/} }

%--------------
% Code Listing
%--------------
%**************************************
% Schrifteinstellungen für Codelistings
%
\setmonofont[Scale=0.75]{Source Code Pro Light}
\definecolor{bg}{rgb}{0.95,0.95,0.95}
\newminted{php}{
	linenos              = true,
	xleftmargin          = 2em,
	tabsize              = 4,
	bgcolor              = bg,
	funcnamehighlighting = true
}

\newminted{mysql}{
	linenos     = true,
	bgcolor     = bg,
	xleftmargin = 2em
}

\usepackage{thesis}
